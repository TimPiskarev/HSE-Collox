\documentclass[12pt,a4paper]{article}
\usepackage[utf8]{inputenc}
\usepackage[russian]{babel}
\usepackage[OT1]{fontenc}
\usepackage{amsmath}
\usepackage{amsfonts}
\usepackage{amssymb}
\title{Коллоквиум по мат. анализу №1}
\begin{document}
\maketitle
\section{Билет}
\begin{itemize}
\item \textbf{Рациональные числа} - числа вида $\frac{p}{q}$, где $q$ - натуральное число, а $p$ - целое. Считается, что две записи $\frac{p_1}{q_1}$ и  $\frac{p_2}{q_2}$ задают одно и то же рациональное число, если $p_1q_2=p_2q_1$. Обратим внимание на то, что рациональных чисел не достаточно для естественных потребностей математики.

\item \textbf{Вещественные числа} - множество всех бесконечно десятичных дробей вида $\pm a_0a_1a_2...$, где $a_0 \in N \vee {0}, a_j \in {0...9}$ (Записи, в которых с какого-то момента стоят только 9-ки запрещены); \\
Число $\pm 0,000...$ называется нулём и совпадает с числом 0;\\
Нунелевое число:
- положительное, если в его записи стоит знак '+';
- отрицательное, если в его записи стоят знак '-'; \\
В вещественные числа вложены рациональные естественным образом. У вещественных чисел также определены операции сложения и умножения для которых справедливы все их естественные свойства. \\
Отношение порядка у вещественных чисел задано лексикографическим порядком. ($a_0a_1a_2...\leq b_0b_1b_2... \exists k: a_0 = b_0, ... a_{k-1}=b_{k-1}, a_k \leq b_k$), который естественным обращом переносится на отрицательные. \\
Для вещественных чисел определён модуль числа $a$, т.е. такое вещественное число, что $|a| = a$, если $a \geq 0$ и $|a| = -a$, если $a < 0$. Также, для модуля выполняется неравенство треугольника $|a+b| \leq |a| + |b|$. Из неравенства треугольника следует, что $||a|-|b|| \leq |a+b|$. \\
Самое важное свойство - выполняется принцип полноты;

\item \textbf{Десятичные дроби.} Рациональное число может быть представлено в виде конечной или периодической десятичной дроби ($\frac{1}{10} = 0.1; \frac{1}{6} = 0.1(6); \frac{1}{7} = 0.(142857)$. Можно не рассматривать десятичные записи с периодом 9, т.к. $0.(9) = 1$ (Если $0.(9) = x$, то $10x=9+x$ - истина, октуда $x = 1$.

\item \textbf{Принцип полноты. }
Принцип полноты выполняется, если для произвольных непустых $A$ левее $B$ найдется разделяющий их элемент. \\
Принцип полноты не выполняется для рациоональных чисел. \\
Принцип полноты выполняется на множестве вещественных чисел (теорема).\\
Доказательство: \\
Пусть $A$ и $B$ - непустые множества.  $A$ левее $B$. Если $A$ состоит только из неположительных чисел, а $B$ только из неоотрицательных, то нуль разделяет $A$ и $B$. Пусть в $A$ имеется положительный элемент, тогда $B$ состоит только из положительных чисел (обратный случай аналогичен). Построим число $c = c_0c_1c_2...$, разделяющее $A$ и $B$. \\
Рассмотрим множество натуральных чисел, с которых начинаются элементы множества $B$. Пусть $b_0$ - наименьшее из таких и пусть $b_0 = c_0$. Затем рассмотрим все числа множестве $B_1$, начинающиеся с $b_0$ и найдем у них наименьшую первую цифру после запятой и предположим, что $b_1=с_1$ (где $b_1$ - эта цифра) и т.д. получим бесконечную десятичную дробь $c_0c_1c_2...$. Покажем, что построенное число рязделяет множества $A$ и $B$. Во-первых, по построению $c \leq b$ для каждого $b \in B$. Действительно, либо $b = c$, либо $b \neq c$. Во втором случае пусть $b_0 = c_0, ..., b_{k-1} = c_{k-1}$ и $b_k \neq c_k$. Тогда по построению числа $c$, $c_k < b_k \Rightarrow c < b$. \\
Покажем, что для каждого $a \in A $ $a \leq c$. Предположим, что $a > c$, т.е. $a \geq c$ и $a \neq c$. Тогда найдется позиция $k$, для которой $a_0=c_0, ..., a_{k-1}=c_{k-1}$ и $a_k > c_k$. Но по построению числа $c$ есть такой $b \in B$, что $b_0 = c_0, b_k=c_k$, значит $a_k > b_k$, что противоречит условию $A$ левее $B$. 

\item Рациональных решений уравнение $x^2=2$ не существует. Действительно, пусть $\frac{p}{q}$ - такое решение и $p$ и $q$ не имеют общих делителей. Тогда $\frac{p^2}{q^2} = 2 \Rightarrow p^2=2q^2 \Rightarrow p^2$ - четное $\Rightarrow$ p - четное $\Rightarrow p=2k, 4k^2=2q^2 \Rightarrow 2k^2 = q^2 \Rightarrow q^2$ - четное $\Rightarrow q$ - четное $\Rightarrow$ числа $p$ и $q$ имеют общий делитель. Противоречие.
\end{itemize}

\section{Билет}
\begin{itemize}
    \item \textbf{Предел последовательности} \\
    Если каждому числу $n \in N$ поставлено в соответствие некоторое число $a_n$, то говорим, что задана числовая последовательность $\{a_n\}_{n=1}^{\infty}$ \\
    Говорят, что последовательность $\{a_n\}_{n=1}^{\infty}$ сходится к числу $a$, если для каждого $\varepsilon > 0$ найдется такой номер $N_{\varepsilon} \in N$, что $|a_n - a| < \varepsilon$ при каждом $n > N_{\varepsilon}$. \\
    $\forall \varepsilon > 0  \exists N_{\varepsilon} \in N : \forall_n> N_{\varepsilon} |a_n - a| < \varepsilon$ \\
    $ \lim_{n\to\infty} a_n \ = a$  или $a_n \to a$ при $n \to \infty$
    \item \textbf{Единственность предела}
    Пусть  $\lim_{n \to \infty} a_n\ = a$ и $\lim_{n \to \infty} b_n\ = b$, тогда a = b. \\
    Доказательство: Если $a \neq b$, то $|a - b| = \varepsilon_0 > 0$. Но по определению найдется номер $N_1$, для которого $|a_n - a| < \frac{\varepsilon_0}{2}$ при $n > N_1$ и найдется номер $N_2$, для которого $|a_n - b| < \frac{\varepsilon_0}{2}$ при $n > N_2$. Тогда при $n > max \{N_1, N_2\} : \varepsilon_0 = |a - b| = |a - a_n + a_n - b| \leq |a - a_n| + |a_n - b| < \varepsilon_0$. Противоречие.
    \item \textbf{Арифметика предела.}
    $\lim_{n \to \infty} a_n\ = a$  $\lim_{n \to \infty} b_n\ = b$\\
    1) $ \lim_{n \to \infty} (\lambda a_n + \beta b_n) = \lambda a + \beta b \;  \forall a, b \in R $ \\
    2)$\lim_{n \to \infty} a_n b_n\ = ab$\\
    3)Если $b \neq 0, b_n \neq 0$, то $\lim_{n \to \infty} \frac{a_n}{b_n}\ = \frac{a}{b}$. \\
    Доказательство: Пусть $\varepsilon > 0$ - произвольное число. Тогда найдется номер $N_1$, для которого $|a_n - a| < \varepsilon$, и найдется номер $N_2$, для которого $|b_n - b| < \varepsilon$\\
    1) При $n > N = max\{ N_1, N_2 \} : |\lambda a_n + \beta b_n - (\lambda a + \beta b)| = |\lambda(a_n - a) + \beta(b_n - b)| \leq |\lambda| |a_n - a| + |\beta| |b_n - b| < (|\lambda| + |\beta|)\varepsilon$ \\
    2) Заметит, что $|a_n b_n - a b| = |a_n b_n - a b_n + a b_n - a b| \leq |b_n| |a_n - a| + |a| |b_n - b|$. Т.к. сходящаяся последовательность ограничена, то найдется $M > 0$, для которого $|b_n| \leq M$, поэтому при  $n > N = max\{ N_1, N_2\}$ выполнено $|a_n b_n - a b| \leq (M + |a|)\varepsilon$\\
    3) Достаточно проверить, что $\frac{1}{b_n} \to \frac{1}{b}$ при $n \to \infty$. Заметим, что по условию $b \neq 0$, поэтому найдется номер $N_3 \in N$, для которого при $n > N_3$ выполнено $|b_n| > \frac{|b|}{2}$. Тогда при $N > max \{N_1, N_2\}$ выполнено $|\frac{1}{b_n} - \frac{1}{b}| = \frac{b_n - b}{|b_n| |b|} \leq \frac{2}{|b|^2} * \varepsilon$\\
    \item \textbf{Ограниченность сходящейся последовательности}:\\
    Утверждение: сходящаяся последовательность ограничена\\
    Доказательство: Если $\lim_{n \to \infty} a_n\ = a$, то для каждого $n \in N$ выполнено  
    $|a_n - a| < 1$ при $n > N  \Rightarrow |a_n| = |a_n - a + a| \leq |a_n - a| + |a| < 1 + |a|$ при $n > N$.
    Значит $|a_n| \leq M = max\{1 + |a|, |a_1|, |a_2|, ..., |a_N|\}$, т. е. $M = c \leq a_n \leq C = M$.\\
    \item \textbf{Определенность}:
    Если $a_n \to a$ и $a \neq 0$, то найдется номер $n \in N$, для которого $|a_n| > \frac{|a|}{2} > 0$ при $n > N$.\\
    Доказательство: Взяв $\varepsilon = \frac{|a|}{2}$ в определении сходимости последовательности к числу $a$, получаем номер $n \in N$, для которого $|a_n - a| < \frac{|a|}{2}$ при $n > N$. Тогда при $n > N$, выполнено $|a| - |a_n| \leq |a_n - a| < \frac{|a|}{2}$, что равносильно тому, что мы доказываем.
  
\end{itemize}
\section{Билет}
\begin{itemize}
\item \textbf{Переход к пределу в неравенствах} \\
Пусть $\lim_{n \to \infty}{a_n = a}, \lim_{n \to \infty}{b_n = b}$. Если для некоторого номера $N$ выполнено $a_n \leq b_n$ при $n > N$, то и $a \leq b$; \\
Доказательство: Предположим, что $a-b = \varepsilon > 0$. Тогда найдутся номера $N_1 \in N$ и $N_2 \in N$, для которых $|a_n - a| < \frac{\varepsilon}{2}$ при $n > N_1$, и $|b_n-b| < \frac{\varepsilon}{2}$ при $n > N_2$. Тогда $\varepsilon = a - b = a - a_n + a_n - b_n + b_n - b \leq a - a_n + b_n - b < \varepsilon$. Противоречие. 

\item \textbf{ Лемма о зажатой последоовательности} \\
Пусть $\lim_{n \to \infty}{a_n = a}$ и $\lim_{n \to \infty}{b_n = b}$ и для некоторого $n \in N$ выполнено неравенство: $a_n \leq c_n \leq b_n$ при $n > N$. Тогда $\lim_{n \to \infty}{c_n = a}$. \\
Доказательство: Для каждого $\varepsilon > 0$ найдутся номера $N_1 \in N$ и $N_2 \in N$, для которых $|a_n - a| < \varepsilon$ и $|b_n - b| < \varepsilon$. Тогда при $n > max\{N, N_1, N_2\}$ выполнено: $a - \varepsilon < a_n \leq c_n \leq b_n < b + \varepsilon$.

\item \textbf{Принцип вложенных отрезков} \\
Пусть $a, b \in R$ и $a < b$. Множества $[a; b] := \{x \in R: a \leq x \leq b\}$, $(a;b) := \{ x \in R: a < x < b \}$ называются отрезком и интервалом соответственно. Длиной отрезка (интервала) называется величина $b - a$. \\
\textbf{Теорема:} Всякая последовательность $\{[a_n, b_n]\}_{n=1}^{\infty}$ вложенных отрезков (т.е. $[a_{n+1};b_{n+1}] \subset [a_{n};b_{n}]$) имеет общую точку. Кроме того, если длины отрезков стремятся к нулю, т.е. $b_n - a_n \longrightarrow 0$, то такая общая точка только одна. \\
Доказательство: по условию $[a_{n+1}; b_{n+1}] \subset [a_{n}; b_{n}]$, откуда $a_{n+1} \leq a_n \leq b_n \leq b_{n+1}$. Заметим, что при $n < m$ выполнено $a_n \leq a_m \leq b_m$, а при $n > m$ выполнено $a_n \leq b_n \leq b_m$. Таким образом, если $A :=\{a_n, n \in N\}$ и $B := \{b_m, m \in N\}$, то $A$ левее $B$, а значит по принципу полноты найдется такое число $c \in R$, что $a_n \leq c \leq b_m$ для произвольных $n, m \in N$, в частности $a_n \leq c \leq b_n$ т.е. $c \in [a_n; b_n]$. \\
Пусть общих точек две: $c$ и $c'$. Не ограничивая ообщности $c < c'$. Тогда $a_n \leq c < c' \leq b_n$ и $c' - c \leq b_n - a_n$, что противоречит тому, что $\lim_{n \to \infty}{(b_n - a_n)} = 0$. Действительно, найдется номер $N$, для которого $b_n - a_n < c' - c$ при каждом $n > N$.

\item \textbf{Геометрическая интерпретация вещественных чисел, вещественная прямая.} \\
Доказанная выше теорема позволяет дать вещественным числам следующую геометрическую интерпретацию. Сопоставим десятичной дроби $0.a_1a_2...$ последовательность вложенных отрезков по следующему правилу: \\
Разделим отрезок $[0;1]$ на 10 равных частей и выберем из получившихся 10 отрезков $(a_1 + 1)$-ый по счету. Делаем то же самое и выбираем $(a_2 + 1)$-ый по счету и т.д. Получаем последовательность вложенных отрезков, причем длина отрезка на $n$-ом шаге равна $10^{-n}$. По доказанной теореме сущестует единственная общая точка построенной последовательности вложенных оторезков, которая как раз и совпадает с $0.a_1a_2...$
\end{itemize}

\section{Билет}
\begin{itemize}
	\item \textbf{Точные верхние и нижние грани.}
	Пусть $A$ - непустое подмножество вещественных чисел.
	Число $b$ называется верхней гранью множества $A$, если $a \leq b$ верно для каждого числа $a \in A$. Если есть хоть одна грань, то множество называют ограниченным сверху. Наименьшая из верхних граней множества $A$ называется точной верхней гранью множества $A$ и обозначается $sup(A)$ (супремум).\\
	Число $b$ называется нижней гранью множества $A$, если $b \leq a$ верно для каждого числа $a \in A$. Ели есть хотя бы одна нижняя грань, то множество называется ограниченным снизу. Наибольшая из нижних граней множества $A$ называется точной нижней гранью множества  $A$ и обозначается $inf(A)$ (инфинум)\\
	Ограниченное и сверху и снизу множество называется ограниченным.
	\item \textbf{Теорема Вейерштрассе о пределе монотонной ограниченной последовательности.} \\
	Пусть последовательность $\{a_n\}_{n=1}^{\infty}$ не убывает $(a_n \leq a_{n + 1})$ и ограничена сверху. Тогда эта последовательность сходится к своему супремуму.\\
	Анологично, пусть поседовательность $\{a_n\}_{n=1}^{\infty}$ не возрастает $(a_{n + 1} \leq a_n)$ и ограничена снизу. Тогда эта последовательность сходится к своему инфинуму.\\
	Доказательство: Пусть $M = sup\{a_n : n \in N\} = sup(a_n)$. Тогда для каждого $\varepsilon > 0$ найдется номер $n \in N$, для которого $M - \varepsilon < a_n, n \in N$ иначе $M - \varepsilon$ - верхняя грань, чего не может быть. В силу того, что последовательность неубывающая, при каждой $n > N$ выполнено $M - \varepsilon < a_N \leq a_n \leq M \leq M + \varepsilon $\\
	Тем самым, по определению $M = \lim\ a_n$\\
	Случай с невозрастающей последовательностью рассматривается аналогично.\\
	\item \textbf{Пример рекуррентной формулы для вычисления $\sqrt2$\\}
	Пусть $a_{n + 1} = \frac{1}{2} (a_n + \frac{2}{a_n}), a_1 = 2$. Заметим, что $a_{n + 1} = \frac{1}{2} (a_n) + \frac{2}{a_n}) \geq \frac{1}{2} (a_n + \frac{2}{a_n})^{\frac{1}{2}} = \sqrt2$\\
	Поэтому $a_n \geq \sqrt2$. Кроме того, $a_{n + 1} = \frac{1}{2} (a_n + \frac{2}{a_n}) \leq \frac{1}{2} (a_n + \frac{a_n}{a_n}) = a_n$. По доказанной Теореме у последовательности $\{a_n\}_{n=1}^{\infty}$ существует предел $a$. Т. к. $a_n \geq 0$, то и $a > 0$.\\
	Тогда по арифметике предела получаем $a = \frac{1}{2} (a + \frac{2}{a})$, откуда $a = \sqrt2$.

\end{itemize}
\end{document}