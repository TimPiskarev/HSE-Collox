\documentclass[12pt,a4paper]{article}
\usepackage[utf8]{inputenc}
\usepackage[russian]{babel}
\usepackage[OT1]{fontenc}
\usepackage{amsmath}
\usepackage{amsfonts}
\usepackage{amssymb}
\title{Коллоквиум по мат. анализу №1}
\begin{document}
\maketitle
\section{Билет}
\begin{itemize}
\item Рациональные числа - числа вида $\frac{p}{q}$, где $q$ - натуральное число, а $p$ - целое. Считается, что две записи $\frac{p_1}{q_1}$ и  $\frac{p_2}{q_2}$ задают одно и то же рациональное число, если $p_1q_2=p_2q_1$. Обратим внимание на то, что рациональных чисел не достаточно для естественных потребностей математики.

\item Вещественные числа - множество всех бесконечно десятичных дробей вида $\pm a_0a_1a_2...$, где $a_0 \in N \vee {0}, a_j \in {0...9}$ (Записи, в которых с какого-то момента стоят только 9-ки запрещены); \\
Число $\pm 0,000...$ называется нулём и совпадает с числом 0;\\
Нунелевое число:
- положительное, если в его записи стоит знак '+';
- отрицательное, если в его записи стоят знак '-'; \\
В вещественные числа вложены рациональные естественным образом. У вещественных чисел также определены операции сложения и умножения для которых справедливы все их естественные свойства. \\
Отношение порядка у вещественных чисел задано лексикографическим порядком. ($a_0a_1a_2...\leq b_0b_1b_2... \exists k: a_0 = b_0, ... a_{k-1}=b_{k-1}, a_k \leq b_k$), который естественным обращом переносится на отрицательные. \\
Для вещественных чисел определён модуль числа $a$, т.е. такое вещественное число, что $|a| = a$, если $a \geq 0$ и $|a| = -a$, если $a < 0$. Также, для модуля выполняется неравенство треугольника $|a+b| \leq |a| + |b|$. Из неравенства треугольника следует, что $||a|-|b|| \leq |a+b|$. \\
Самое важное свойство - выполняется принцип полноты;

\item Десятичные дроби. Рациональное число может быть представлено в виде конечной или периодической десятичной дроби ($\frac{1}{10} = 0.1; \frac{1}{6} = 0.1(6); \frac{1}{7} = 0.(142857)$. Можно не рассматривать десятичные записи с периодом 9, т.к. $0.(9) = 1$ (Если $0.(9) = x$, то $10x=9+x$ - истина, октуда $x = 1$.

\item Принцип полноты. 
Принцип полноты выполняется, если для произвольных непустых $A$ левее $B$ найдется разделяющий их элемент. \\
Принцип полноты не выполняется для рациоональных чисел. \\
Принцип полноты выполняется на множестве вещественных чисел (теорема).\\
Доказательство: \\
Пусть $A$ и $B$ - непустые множества.  $A$ левее $B$. Если $A$ состоит только из неположительных чисел, а $B$ только из неоотрицательных, то нуль разделяем на $A$ и $B$. Пусть в $A$ имеется положительный элемент, тогда $B$ состоит только из положительных чисел (обратный случай аналогичен). Построим число $c = c_0c_1c_2...$, разделяющее $A$ и $B$. \\
Рассмотрим множество натуральных чисел, с которых начинаются элементы множества $B$. Пусть $b_0$ - наименьшее


\end{itemize}

\section{Билет}
\begin{itemize}
    \item Предел последовательности \\
    Если каждому числу $n \in N$ поставлено в соответствие некоторое число $a_n$, то говорим, что задана числовая последовательность $\{a_n\}_{n=1}^{\infty}$ \\
    Говорят, что последовательность $\{a_n\}_{n=1}^{\infty}$ сходится к числу $a$, если для каждого $\varepsilon > 0$ найдется такой номер $N_{\varepsilon} \in N$, что $|a_n - a| < \varepsilon$ при каждом $n > N_{\varepsilon}$. \\
    $\forall \varepsilon > 0  \exists N_{\varepsilon} \in N : \forall_n> N_{\varepsilon} |a_n - a| < \varepsilon$ \\
    $ \lim_{n\to\infty} a_n \ = a$  или $a_n \to a$ при $n \to \infty$
    \item Единственность предела
    Пусть  $\lim_{n \to \infty} a_n\ = a$ и $\lim_{n \to \infty} b_n\ = b$, тогда a = b. \\
    Доказательство: Если $a \neq b$, то $|a - b| = \varepsilon_0 > 0$. Но по определению найдется номер $N_1$, для которого $|a_n - a| < \frac{\varepsilon_0}{2}$ при $n > N_1$ и найдется номер $N_2$, для которого $|a_n - b| < \frac{\varepsilon_0}{2}$ при $n > N_2$. Тогда при $n > max \{N_1, N_2\} : \varepsilon_0 = |a - b| = |a - a_n + a_n - b| \leq |a - a_n| + |a_n - b| < \varepsilon_0$. Противоречие.
    \item Арифметика предела.
    $\lim_{n \to \infty} a_n\ = a$  $\lim_{n \to \infty} b_n\ = b$\\
    1) $ \lim_{n \to \infty} (\lambda a_n + \beta b_n) = \lambda a + \beta b \;  \forall a, b \in R $ \\
    2)$\lim_{n \to \infty} a_n b_n\ = ab$\\
    3)Если $b \neq 0, b_n \neq 0$, то $\lim_{n \to \infty} \frac{a_n}{b_n}\ = \frac{a}{b}$. \\
    Доказательство: Пусть $\varepsilon > 0$ - произвольное число. Тогда найдется номер $N_1$, для которого $|a_n - a| < \varepsilon$, и найдется номер $N_2$, для которого $|b_n - b| < \varepsilon$\\
    1) При $n > N = max\{ N_1, N_2 \} : |\lambda a_n + \beta b_n - (\lambda a + \beta b)| = |\lambda(a_n - a) + \beta(b_n - b)| \leq |\lambda| |a_n - a| + |\beta| |b_n - b| < (|\lambda| + |\beta|)\varepsilon$ \\
    2) Заметит, что $|a_n b_n - a b| = |a_n b_n - a b_n + a b_n - a b| \leq |b_n| |a_n - a| + |a| |b_n - b|$. Т.к. сходящаяся последовательность ограничена, то найдется $M > 0$, для которого $|b_n| \leq M$, поэтому при  $n > N = max\{ N_1, N_2\}$ выполнено $|a_n b_n - a b| \leq (M + |a|)\varepsilon$\\
    3) Достаточно проверить, что $\frac{1}{b_n} \to \frac{1}{b}$ при $n \to \infty$. Заметим, что по условию $b \neq 0$, поэтому найдется номер $N_3 \in N$, для которого при $n > N_3$ выполнено $|b_n| > \frac{|b|}{2}$. Тогда при $N > max \{N_1, N_2\}$ выполнено $|\frac{1}{b_n} - \frac{1}{b}| = \frac{b_n - b}{|b_n| |b|} \leq \frac{2}{|b|^2} * \varepsilon$\\
    \item Ограниченность сходящейся последовательности:\\
    Утверждение: сходящаяся последовательность ограничена\\
    Доказательство: Если $\lim_{n \to \infty} a_n\ = a$, то для каждого $n \in N$ выполнено  
    $|a_n - a| < 1$ при $n > N  \Rightarrow |a_n| = |a_n - a + a| \leq |a_n - a| + |a| < 1 + |a|$ при $n > N$.
    Значит $|a_n| \leq M = max\{1 + |a|, |a_1|, |a_2|, ..., |a_N|\}$, т. е. $M = c \leq a_n \leq C = M$.\\
    \item Определенность:
    Если $a_n \to a$ и $a \neq 0$, то найдется номер $n \in N$, для которого $|a_n| > \frac{|a|}{2} > 0$ при $n > N$.\\
    Доказательство: Взяв $\varepsilon = \frac{|a|}{2}$ в определении сходимости последовательности к числу $a$, получаем номер $n \in N$, для которого $|a_n - a| < \frac{|a|}{2}$ при $n > N$. Тогда при $n > N$, выполнено $|a| - |a_n| \leq |a_n - a| < \frac{|a|}{2}$, что равносильно тому, что мы доказываем.
  
\end{itemize}


\section{Билет}
\begin{itemize}
	\item Точные верхние и нижние границы
\end{itemize}
\end{document}